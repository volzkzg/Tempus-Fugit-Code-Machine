\documentclass[landscape, twocolumn, 8pt, a4paper, twoside]{extarticle}



\usepackage{mathrsfs} %支持花体字母
\usepackage{bm, amsmath, amsthm, graphicx, amsfonts, fancyhdr, ctable, pict2e, multirow, rotating, geometry, listings, color, fontspec, geometry}
\usepackage{xeCJK}
\geometry{left=0.5cm, right=0.5cm, top=1.5cm, bottom=1.5cm}
\setCJKmainfont{PingFangSC-Light}
\newfontfamily{\lstsansserif}[Scale=0.70]{Monaco}

\definecolor{base03}{RGB}{0,43,54}
\definecolor{base02}{RGB}{7,54,66}
\definecolor{base01}{RGB}{88,110,117}
\definecolor{base00}{RGB}{101,123,131}
\definecolor{base0}{RGB}{131,148,150}
\definecolor{base1}{RGB}{147,161,161}
\definecolor{base2}{RGB}{238,232,213}
\definecolor{base3}{RGB}{253,246,227}
\definecolor{YELLOW}{RGB}{181,137,0}
\definecolor{ORANGE}{RGB}{203,75,22}
\definecolor{RED}{RGB}{220,50,47}
\definecolor{MAGENTA}{RGB}{211,54,130}
\definecolor{VIOLET}{RGB}{108,113,196}
\definecolor{BLUE}{RGB}{38,139,210}
\definecolor{CYAN}{RGB}{42,161,152}
\definecolor{GREEN}{RGB}{133,153,0}
\definecolor{codegreen}{rgb}{0,0.6,0}
\definecolor{codegray}{rgb}{0.5,0.5,0.5}
\definecolor{codepurple}{rgb}{0.58,0,0.82}
\definecolor{backcolour}{rgb}{0.95,0.95,0.92}
 
\lstdefinestyle{mystyle}{
  language=C++,
  sensitive=true,
    %backgroundcolor=\color{base3},   
    basicstyle=\color{CYAN}\lstsansserif,
    commentstyle=\color{RED},
    keywordstyle=\color{BLUE},
    numberstyle=\tiny\color{VIOLET},
    stringstyle=\color{YELLOW},
    identifierstyle=\color{base00},
    breakatwhitespace=false,         
    breaklines=true,                 
    captionpos=b,                    
    keepspaces=true,                 
    numbers=left,                    
    numbersep=5pt,                  
    showspaces=false,                
    showstringspaces=false,
    showtabs=false,                  
    tabsize=4,
    xleftmargin=2.0em,
    xrightmargin=2.0em
}
 
\lstset{style=mystyle}
\pagestyle{fancy}
\lhead{Shanghai Jiao Tong University}
\chead{\thepage}
\rhead{Tempus Fugit}
\lfoot{}
\cfoot{}
\rfoot{}
\newcommand{\stlf}[2]{\genfrac{ [ }{ ] }{0pt}{}{#1}{#2}}
\newcommand{\stls}[2]{\genfrac{ \{ }{ \} }{0pt}{}{#1}{#2}}

\begin{document}

\tableofcontents
\newpage
\section{计算几何}
  \subsection{二维计算几何基本操作}
  \lstinputlisting{"Computational Geometry/2D_Geometry_Base.cpp"}
  
  \subsection{二维计算几何基本操作}
  \lstinputlisting{"Computational Geometry/2D_Geometry_Base.cpp"}

  \subsection{圆的面积模板}
  \lstinputlisting{"Computational Geometry/CircleKCover.cpp"}

  \subsection{多边形相关}
  \lstinputlisting{"Computational Geometry/Polygon.cpp"}

  \subsection{直线与凸包求交点}
  \lstinputlisting{"Computational Geometry/isPL.cpp"}

  \subsection{半平面交}
  \lstinputlisting{"Computational Geometry/Half Plane Intersection.cpp"}

  \subsection{最大面积空凸包}
  \lstinputlisting{"Computational Geometry/Max Area Empty Convex Hull.cpp"}

  \subsection{最近点对}
  \lstinputlisting{"Computational Geometry/ClosestPair.cpp"}

  \subsection{凸包与点集直径}
  \lstinputlisting{"Computational Geometry/Convex Hull && Convex Diameter.cpp"}

  \subsection{Farmland}
  \lstinputlisting{"Computational Geometry/Farmland.cpp"}

  \subsection{Voronoi图}
  不能有重点, 点数应当不小于2
  \lstinputlisting{"Computational Geometry/Voronoi.cpp"}

  \subsection{四边形双费马点}
  \lstinputlisting{"Computational Geometry/Fermat Point Of A Quadrilateral.cpp"}

  \subsection{三角形和四边形的费马点}
  \begin{itemize}
  \item 费马点: 距几个顶点距离之和最小的点
  \item 三角形:
      \begin{itemize}
      \item 若每个角都小于 $120^{\circ}$: 以每条边向外作正三角形, 得到 $\Delta ABF$, $\Delta BCD$, $\Delta CAE$, 连接$AD$, $BE$, $CF$, 三线必共点于费马点. 该点对三边的张角必然是$120^{\circ}$, 也必然是三个三角形外接圆的交点
      \item 否则费马点一定是那个大于等于$120^{\circ}$的顶角
      \end{itemize}
  \item 四边形:
      \begin{itemize}
        \item 在凸四边形中, 费马点为对角线的交点
        \item 在凹四边形中, 费马点位凹顶点
      \end{itemize}
  \end{itemize}

  \subsection{三维计算几何基本操作}
  \lstinputlisting{"Computational Geometry/3D_Geometry_Base.cpp"}

  \subsection{凸多面体切割}
  \lstinputlisting{"Computational Geometry/3D Convex Cut.cpp"}

  \subsection{三维凸包}
  不能有重点
  \lstinputlisting{"Computational Geometry/3D Convex Hull.cpp"}

  \subsection{球面点表面点距离}
  \lstinputlisting{"Computational Geometry/DistOnBall.cpp"}

  \subsection{长方体表面点距离}
  \lstinputlisting{"Computational Geometry/DistOnCuboid.cpp"}

  \subsection{最小覆盖球}
  \lstinputlisting{"Computational Geometry/MinBall.cpp"}

  \subsection{三维向量操作矩阵}
  \begin{itemize}
  \item 绕单位向量$u = (u_x, u_y, u_z)$右手方向旋转$\theta$度的矩阵: \\
    $
    \begin{bmatrix}
      \cos\theta + u_x^2 (1 - \cos\theta)    &    u_x u_y (1 - \cos\theta) - u_z \sin\theta    &    u_x u_z (1 - \cos\theta) + u_y \sin\theta \\
    u_y u_x (1 - \cos\theta) + u_z \sin\theta    &    \cos\theta + u_y^2 (1 - \cos\theta)    &    u_y u_z (1 - \cos\theta) - u_x \sin\theta \\
    u_z u_x (1 - \cos\theta) - u_y \sin\theta    &    u_z u_y (1 - \cos\theta) + u_x \sin\theta    &    \cos\theta + u_z^2 (1 - \cos\theta)
  \end{bmatrix} \\
  = \cos\theta I
  + \sin\theta
  \begin{bmatrix}
    0    &    -u_z    &    u_y \\
    u_z    &    0    &    -u_x \\
    -u_y    &    u_x    &    0
  \end{bmatrix}
  + (1 - \cos\theta)
  \begin{bmatrix}
    u_x ^ 2    &    u_x u_y    &    u_x u_z \\
    u_y u_x    &    u_y ^ 2    &    u_y u_z \\
    u_z u_x    &    u_z u_y    &    u_z ^ 2
  \end{bmatrix}
  $
  \item 点$a$绕单位向量$u = (u_x, u_y, u_z)$右手方向旋转$\theta$度的对应点为
  $a^\prime = a \cos\theta + (u \times a) \sin\theta + (u \otimes u) a (1 - \cos\theta)$
  \item 关于向量 $v$ 作对称变换的矩阵 $H = I - 2 \frac{v v^T}{v^T v}$,
  \item 点$a$对称点: $a^\prime = a - 2 \frac{v^T a}{v^T v} \cdot v$
  \end{itemize}
  \subsection{立体角}
  对于任意一个四面体$OABC$,
  从$O$点观察$\Delta ABC$的立体角$\tan{\frac{\Omega}{2}} = 
  \frac{\textrm{mix}(\overrightarrow{a}, \overrightarrow{b}, \overrightarrow{c}) }{
  |a||b||c|
  + (\overrightarrow{a} \cdot \overrightarrow{b}) |c|
  + (\overrightarrow{a} \cdot \overrightarrow{c}) |b|
  + (\overrightarrow{b} \cdot \overrightarrow{c}) |a|
}$.

\section{数据结构}
\subsection{动态凸包(只支持插入)}
\lstinputlisting{"Data Structure/Dynamic Convex Hull (Insertion Only).cpp"}

\subsection{Rope用法}
\lstinputlisting{"Data Structure/Usage Of Rope.cpp"}

\subsection{Treap}
\lstinputlisting{"Data Structure/Treap.cpp"}

\subsection{可持久化Treap}
\lstinputlisting{"Data Structure/Functional Treap.cpp"}

\subsection{左偏树}
\lstinputlisting{"Data Structure/Leftist Heap.cpp"}

\subsection{Link-Cut Tree}
\lstinputlisting{"Data Structure/Link Cut Tree.cpp"}

\subsection{K-D Tree Nearest}
\lstinputlisting{"Data Structure/KDTree.cpp"}

\subsection{K-D Tree Farthest}
输入 $n$ 个点, 对每个询问 $px, py, k$, 输出 $k$ 远点的编号
\lstinputlisting{"Data Structure/KDTree_2.cpp"}

\subsection{K-D Tree Beautiful}
\lstinputlisting{"Data Structure/kd-tree.cpp"}

\subsection{树链剖分}
\lstinputlisting{"Data Structure/HeavyLightTree.cpp"}

\subsection{Splay维护数列}
\lstinputlisting{"Data Structure/Splay_Array.cpp"}

\section{字符串相关}
\subsection{Manacher}
\lstinputlisting{"String Related/Manacher.cpp"}

\subsection{最大回文正方形}
\lstinputlisting{"String Related/Palindrome_Square.cpp"}

\subsection{KMP}
$next[i] = \max\{len | A[0 \ldots len - 1] = A \textrm{的第i位向前或后的长度为 $len$ 的串} \}$

$ext[i] = \max\{len | A[0 \ldots len - 1] = B \textrm{的第i位向前或后的长度为 $len$ 的串} \}$
\lstinputlisting{"String Related/KMP.cpp"}

\subsection{Aho-Corasick 自动机}
\lstinputlisting{"String Related/Aho Corasick Automaton.cpp"}

\subsection{后缀自动机}
\lstinputlisting{"String Related/Suffix Automaton.cpp"}

\subsection{后缀数组-1}
待排序的字符串放在 $r[0 \ldots n - 1]$ 中, 最大值小于 $m$.

$r[0 \ldots n - 2] > 0$, $r[n - 1] = 0$.

结果放在 $sa[0 \ldots n - 1]$.
\lstinputlisting{"String Related/Suffix Array Backup.cpp"}

\subsection{后缀数组-2}
\lstinputlisting{"String Related/Suffix Array.cpp"}

\subsection{环串最小表示}
\lstinputlisting{"String Related/Minimial Representation Of A Cyclic String.cpp"}
\subsection{回文自动机}
\lstinputlisting{"String Related/PA.cpp"}

\section{图论}
\subsection{Dominator Tree}
\lstinputlisting{"Graph Theory/DominatorTree.cpp"}

\subsection{树 Hash}
\lstinputlisting{"Graph Theory/TreeHash.cpp"}

\subsection{带花树}
\lstinputlisting{"Graph Theory/Blossom Algorithm.cpp"}

\subsection{最大流}
\lstinputlisting{"Graph Theory/Maxflow.cpp"}

\subsection{最高标号预流推进}
\lstinputlisting{"Graph Theory/Highest Level Preflow-Push.cpp"}

\subsection{KM}
\lstinputlisting{"Graph Theory/KM.cpp"}

\subsection{双连通分量}
\lstinputlisting{"Graph Theory/BCC.cpp"}

\subsection{强连通分量}
\lstinputlisting{"Graph Theory/SCC.cpp"}

\subsection{2-SAT 与 Kosaraju}
注意 Kosaraju 需要建反图
\lstinputlisting{"Graph Theory/TwoSat SCC.cpp"}

\subsection{全局最小割 Stoer-Wagner}
\lstinputlisting{"Graph Theory/Stoer-Wagner.cpp"}

\subsection{Hopcroft-Karp}
\lstinputlisting{"Graph Theory/HopcroftKarp.cpp"}

\subsection{欧拉路}
\lstinputlisting{"Graph Theory/EulerianWalk.cpp"}

\subsection{稳定婚姻}
\lstinputlisting{"Graph Theory/StableMatching.cpp"}

\subsection{最大团搜索}
\lstinputlisting{"Graph Theory/MaxClique.cpp"}

\subsection{极大团计数}
\lstinputlisting{"Graph Theory/MaxCliqueCounting.cpp"}

\subsection{最小树形图}
\lstinputlisting{"Graph Theory/Minimum Optimum branchings.cpp"}

\subsection{离线动态最小生成树}
$O (Q log^2 Q) $.
$(qx[i], qy[i])$ 表示将编号为 $qx[i]$ 的边的权值改为 $qy[i]$,
删除一条边相当于将其权值改为 $\infty$,
加入一条边相当于将其权值从 $\infty$ 变成某个值.
\lstinputlisting{"Graph Theory/Dynamic MST(Offline).cpp"}

\subsection{弦图}
\begin{itemize}
  % \item 团数 $\le$ 色数
  % \item 最大独立集数 $\le$ 最小团覆盖数
\item 任何一个弦图都至少有一个单纯点, 不是完全图的弦图至少有两个不相邻的单纯点. 
\item 设第i个点在弦图的完美消除序列第 $p(i)$个. 令 $N(v) = \{w | w \text{与} v \text{相邻且} p(w) > p(v) \}$弦图的极大团一定是 $v \cup N(v)$ 的形式. 
\item 弦图最多有$n$个极大团. 
\item 设 $next(v)$ 表示 $N(v)$中最前的点. 令 $w*$ 表示所有满足 $A\in B$ 的 $w$ 中最后的一个点. 
  判断 $v \cup N(v)$是否为极大团,
  只需判断是否存在一个 $w$, 
  满足 $Next(w) = v$ 且 $|N(v)| + 1 \le |N(w)|$ 即可. 
\item 最小染色:完美消除序列从后往前依次给每个点染色, 给每个点染上可以染的最小的颜色. (团数 = 色数)
\item 最大独立集:完美消除序列从前往后能选就选. 
\item 最小团覆盖:设最大独立集为 $\{p_1, p_2, \ldots, p_t\}$, 则 $\{p_1 \cup N(p_1), \ldots, p_t \cup N(p_t) \}$为最小团覆盖.  (最大独立集数 = 最小团覆盖数)
\end{itemize}

\lstinputlisting{"Graph Theory/MCS.cpp"}

\subsection{K短路(允许重复)}
\lstinputlisting{"Graph Theory/K-shortest Path (Repeat is allowed).cpp"}

\subsection{K短路(不允许重复)}
\lstinputlisting{"Graph Theory/K-shortest Path (Repeat is not allowed).cpp"}

\subsection{小知识}
\begin{itemize}
\item 平面图: 一定存在一个度小于等于$5$的点. $E \le 3V - 6$. 欧拉公式: $V + F - E = 1 + \mbox{连通块数}$
\item 图连通度: 
  \begin{enumerate}
  \item $k-$连通(\emph{k-connected}): 对于任意一对结点都至少存在结点各不相同的$k$条路
  \item 点连通度(\emph{vertex connectivity}): 把图变成非连通图所需删除的最少点数
  \item Whitney定理: 一个图是$k-$连通的当且仅当它的点连通度至少为$k$
  \end{enumerate}
\item Lindstroem-Gessel-Viennot Lemma:
  给定一个图的$n$个起点和$n$个终点, 
  令$A_{ij} = $第$i$个起点到第$j$个终点的路径条数,
  则从起点到终点的不相交路径条数为 $det(A)$
\item 欧拉回路与树形图的联系: 
  对于出度等于入度的连通图
  $s(G) = t_i(G) \prod_{j = 1}^{n} (d^+(v_j) - 1)! $
\item 密度子图: 给定无向图, 选取点集及其导出子图, 最大化 $W_e + P_v$ (点权可负).
  \begin{itemize}
  \item $(S, u) = U$, $(u, T) = U - 2 P_u - D_u$, $(u, v) = (v, u) = W_e$
  \item ans = $\frac{Un - C[S, T]}{2}$, 解集为 $S - \{s\}$
  \end{itemize}
\item 最大权闭合图: 选$a$则$a$的后继必须被选
  \begin{itemize}
  \item $P_u > 0$, $(S, u) = P_u$, $P_u < 0$, $(u, T) = -P_u$
  \item ans = $\sum\limits_{P_u > 0}^{} P_u - C[S, T]$, 解集为 $S - \{s\}$
  \end{itemize}
\item 判定边是否属于最小割:
  \begin{itemize}
  \item 可能属于最小割: $(u, v)$ 不属于同一SCC
  \item 一定在所有最小割中: $(u, v)$ 不属于同一SCC, 且 $S, u$ 在同一 SCC, $u, T$ 在同一SCC
  \end{itemize}
\item 图同构 Hash: $F_t(i) = (F_{t - 1}(i) \times A
  + \sum_{i \rightarrow j} F_{t - 1}(j) \times B
  + \sum_{j \leftarrow i} F_{t - 1}(j) \times C
  + D \times (i = a)
  ) \pmod{P}$, 枚举点$a$, 迭代$K$次后求得的$F_k(a)$就是$a$点所对应的 Hash 值. 
  % $\arcsin x = x + \sum_{n = 1}^{\infty} \frac{(2n - 1)!!}{(2n)!!} \frac{x^{2n + 1}}{2n + 1}$
\end{itemize}

\section{数学}
\subsection{博弈论相关}
\begin{enumerate}
	\item Anti-SG:\\
		规则与Nim基本相同,取最后一个的输。\\
		先手必胜当且仅当:\\
		(1) 所有堆的石子数都为1且游戏的SG值为0;\\
		(2) 有些堆的石子数大于1且游戏的SG值不为0。\\
	\item SJ定理:\\
		对于任意一个Anti-SG游戏,如果我们规定当局面中,所有的单一游戏的SG值为0时,游戏结束,则先手必胜当且仅当:\\
		(1) 游戏的SG函数不为0且游戏中某个单一游戏的SG函数大于1;\\
		(2) 游戏的SG函数为0且游戏中没有单一游戏的SG函数大于1。\\
	\item Multi-SG游戏:\\
		可以将一堆石子分成多堆.\\
	\item Every-SG游戏:\\
		每一个可以移动的棋子都要移动.\\
		对于我们可以赢的单一游戏,我们一定要拿到这一场游戏的胜利.\\
		只需要考虑如何让我们必胜的游戏尽可能长的玩下去,对手相反。\\
		于是就来一个DP,\\
		step[v] = 0;(v为终止状态)\\
		step[v] = max{step[u]} + 1;(sg[v]>0,sg[u]=0)\\
		step[v] = min{step[u]} + 1;(sg[v]==0)\\
	\item 翻硬币游戏:\\
		N枚硬币排成一排,有的正面朝上,有的反面朝上。游戏者根据某些约束翻硬币(如:每次只能翻一或两枚,或者每 次只能翻连续的几枚),但他所翻动的硬币中,最右边的必须是从正面翻到反面。谁不能翻谁输。\\
		结论:局面的SG值为局面中每个正面朝上的棋子单一存在时的SG值的异或和。可用数学归纳法证明。\\
	\item 无向树删边游戏:\\
		规则如下:\\
		给出一个有N个点的树,有一个点作为树的根节点。游戏者轮流从树中删去边,删去一条边后,不与根节点相连的部分将被移走。谁无路可走谁输。\\
		结论:\\
		叶子节点的SG值为0;中间节点的SG值为它的所有子节点的SG值加1后的异或和。是用数学归纳法证明。\\
	\item Christmas Game(PKU3710):\\
		题目大意:\\
		有N个局部联通的图。Harry和Sally轮流从图中删边,删去一条边后,不与根节点相连的部分将被移走。Sally为先手。图是通过从基础树中加一些边得到的。所有形成的环保证不共用边,且只与基础树有一个公共点。谁无路可走谁输。环的处理成为了解题的关键。\\
		性质:\\
		(1)对于长度为奇数的环,去掉其中任意一个边之后,剩下的两个链长度同奇偶,抑或之后的SG值不可能为奇数,所以它的SG值为1;\\
		(2)对于长度为偶数的环,去掉其中任意一个边之后,剩下的两个链长度异奇偶,抑或之后的SG值不可能为0,所以它的SG值为0;所以我们可以去掉所有的偶环,将所有的奇环变为长短为1的链。\\
		这样的话,我们已经将这道题改造成了上一节的模型。\\
	\item 无向图的删边游戏:\\
		我们将Christmas Game这道题进行一步拓展——去掉对环的限制条件,这个模型应该怎样处理?\\
		无向图的删边游戏:\\
		一个无向联通图,有一个点作为图的根。游戏者轮流从图中删去边,删去一条边后,不与根节点相连的部 分将被移走。谁无路可走谁输。 \\
		结论:\\
		对无向图做如下改动:将图中的任意一个偶环缩成一个新点,任意一个奇环缩成一个新点加一个新边;所有连到原先环上的边全部改为与新点相连。这样的改动不会影响图的SG值。\\
	\item Staircase nim:\\
		楼梯从地面由下向上编号为0到n。游戏者在每次操作时可以将楼梯j(1<=j<=n)上的任意多但至少一个硬币移动到楼梯j-1上。将最后一枚硬币移至地上的人获胜。\\
		结论:\\
		设该游戏Sg函数为奇数格棋子数的Xor和S。\\
		如果S=0,则先手必败,否则必胜。\\
\end{enumerate}

\subsection{单纯形Cpp}
max $\left \{ cx | Ax \le b, x \ge 0 \right \}$
\lstinputlisting{"Math/SimplexCpp.cpp"}

\subsection{单纯形Java}
% max $\left \{ cx | Ax \le b, x \ge 0 \right \}$
\lstinputlisting[language=JAVA]{"Math/SimplexJava.java"}

\subsection{自适应辛普森}
\lstinputlisting{"Math/AdaptiveSimpson.cpp"}

\subsection{高斯消元}
\lstinputlisting{"Math/GaussElimination.cpp"}

\subsection{FFT}
\lstinputlisting{"Math/FFT.cpp"}

\subsection{整数FFT}
\lstinputlisting{"Math/FFT_Integer.cpp"}

\subsection{扩展欧几里得}
$ax + by = g = gcd(x, y)$
\lstinputlisting{"Math/ExGcd.cpp"}

\subsection{线性同余方程}
\begin{itemize}
\item 中国剩余定理:
  设$m_1, m_2, \cdots, m_k$ 两两互素, 则同余方程组 $x \equiv a_i \pmod{m_i} \textrm{for $i = 1, 2, \cdots, k$}$
  在$[0, M = m_1 m_2 \cdots m_k)$内有唯一解. 
  记 $M_i = M / m_i$,
  找出 $p_i$ 使得 $M_i p_i \equiv 1 \pmod{m_i}$,
  记 $e_i = M_i p_i$,
  则 $x \equiv e_1 a_1 + e_2 a_2 + \cdots + e_k a_k \pmod{M}$
\item 多变元线性同余方程组:
  方程的形式为 $a_1 x_1 + a_2 x_2 + \cdots + a_n x_n + b \equiv 0 \pmod{m}$,
  令 $d = (a_1, a_2, \cdots, a_n, m)$,
  有解的充要条件是 $d | b$, 解的个数为 $m^{n - 1} d$
\end{itemize}

\subsection{Miller-Rabin 素性测试}
\lstinputlisting{"Math/MillerRabin.cpp"}

\subsection{PollardRho}
\lstinputlisting{"Math/PollardRho.cpp"}

\subsection{多项式求根}
\lstinputlisting{"Math/Root Of Polynomial.cpp"}

\subsection{线性递推}
for $a_{i + n} = (\sum_{i = 0}^{n - 1} k_j a_{i + j}) + d$,
$a_m = (\sum_{i = 0}^{n - 1} c_i a_i) + c_n d$
\lstinputlisting{"Math/Linear Recursive Equation.cpp"}

\subsection{原根}
原根 $g$: $g$ 是模 $n$ 简化剩余系构成的乘法群的生成元.
模 $n$ 有原根的充要条件是 $n = 2, 4, p^n, 2p^n$, 其中 $p$ 是奇质数, $n$ 是正整数
\lstinputlisting{"Math/PrimitiveRoot.cpp"}

\subsection{离散对数}
$A ^ x \equiv B \pmod{C}$, 对非质数 $C$ 也适用.
\lstinputlisting{"Math/ModLog.cpp"}

\subsection{平方剩余}
\begin{itemize}
\item Legrendre Symbol: 对奇质数 $p$, $(\frac{a}{p}) =
  \left\{
    \begin{aligned}
      &  1 & \textrm{是平方剩余} \\
      & -1 & \textrm{是非平方剩余} \\
      &  0 & \textrm{$a \equiv 0 \pmod{p}$}
    \end{aligned}
  \right.
  = a^{\frac{p - 1}{2}} \bmod{p} $
\item 若 $p$ 是奇质数, $(\frac{-1}{p}) = 1$ 当且仅当 $p \equiv 1 \pmod{4}$
\item 若 $p$ 是奇质数, $(\frac{ 2}{p}) = 1$ 当且仅当 $p \equiv \pm 1 \pmod{8}$
\item 若 $p, q$ 是奇素数且互质, $(\frac{p}{q})(\frac{q}{p}) = (-1)^{\frac{p - 1}{2} \times \frac{q - 1}{2}}$
\item Jacobi Symbol: 对奇数 $n = p_1^{\alpha_1} p_2^{\alpha_2} \cdots p_k ^ {\alpha_k} $, 
  $(\frac{a}{n}) = (\frac{a}{p_1})^{\alpha_1} (\frac{a}{p_2})^{\alpha_2} \cdots (\frac{a}{p_k})^{\alpha_k}$
\item Jacobi Symbol 为 $-1$ 则一定不是平方剩余, 所有平方剩余的 Jacobi Symbol 都是 1, 但1不一定是平方剩余
\end{itemize}

$ax^2 + bx + c \equiv 0 \pmod{p}$, 其中 $a \ne 0 \pmod{p}$, 且 $p$ 是质数
\lstinputlisting{"Math/QuadraticResidue.cpp"}

\subsection{N次剩余}
\begin{itemize}
\item 若$p$为奇质数, $a$为$p$的$n$次剩余的充要条件是$a^{\frac{p - 1}{(a, p - 1)}} \equiv 1 \pmod{p}$.
\end{itemize}
$x^N \equiv a \pmod{p}$, 其中$p$是质数
\lstinputlisting{"Math/Nth Residue.cpp"}

\subsection{Pell方程}
$\begin{pmatrix}
  x_k \\ y_k
\end{pmatrix} = 
\begin{pmatrix}
  x_1 & dy_1 \\
  y_1 & x_1
\end{pmatrix} ^ {k - 1}
\begin{pmatrix}
  x_1 \\ y_1
\end{pmatrix}
$

\lstinputlisting{"Math/Pell.cpp"}

\subsection{Romberg积分}
\lstinputlisting{"Math/Romberg.cpp"}

\subsection{公式}
\subsubsection{级数与三角}
\begin{itemize}
\item
  $\sum_{k=1}^{n}k^3 = (\frac{n(n+1)}{2})^2  $
\item
  $\sum_{k=1}^{n}k^4 = \frac{n(n+1)(2n+1)(3n^2+3n-1)}{30}  $
\item
  $\sum_{k=1}^{n}k^5 = \frac{n^2(n+1)^2(2n^2+2n-1)}{12}  $
\item
  $\sum_{k=1}^{n}k(k+1) = \frac{n(n+1)(n+2)}{3}  $
\item
  $\sum_{k=1}^{n}k(k+1)(k+2) = \frac{n(n+1)(n+2)(n+3)}{4} $
\item
  $\sum_{k=1}^{n}k(k+1)(k+2)(k+3) = \frac{n(n+1)(n+2)(n+3)(n+4)}{5} $
\item 错排:
  $D_n = n!(1-\frac{1}{1!}+\frac{1}{2!}-\frac{1}{3!}+\ldots+\frac{(-1)^n}{n!}) = (n-1)(D_{n-2}-D_{n-1})$
\item
  $\sin(\alpha \pm \beta) = \sin\alpha\cos\beta \pm \cos\alpha\sin\beta $
\item
  $\cos(\alpha \pm \beta) = \cos\alpha\cos\beta \mp \sin\alpha\sin\beta $
\item
  $\tan(\alpha \pm \beta) = \frac{\tan\alpha \pm \tan\beta}{1 \mp \tan\alpha\tan\beta} $
\item
  $\tan\alpha \pm \tan\beta = \frac{\sin(\alpha \pm \beta)}{\cos\alpha\cos\beta} $
\item
  $\sin\alpha+\sin\beta = 2\sin\frac{\alpha+\beta}{2}\cos\frac{\alpha-\beta}{2} $
\item
  $\sin\alpha-\sin\beta = 2\cos\frac{\alpha+\beta}{2}\sin\frac{\alpha-\beta}{2} $
\item
  $\cos\alpha+\cos\alpha = 2\cos\frac{\alpha+\beta}{2}\cos\frac{\alpha-\beta}{2} $
\item
  $\cos\alpha-\cos\beta = -2\sin\frac{\alpha+\beta}{2}\sin\frac{\alpha-\beta}{2} $
\item
  $\cos{n\alpha} =
  \binom{n}{0} \cos ^ {n} \alpha
  - \binom{n}{2} \cos ^ {n - 2} \alpha \sin ^ {2} \alpha
  + \binom{n}{4} \cos ^ {n - 4} \alpha \sin ^ {4} \alpha
  \cdots $
\item
  $\sin{n\alpha} =
  \binom{n}{1} \cos ^ {n - 1} \alpha \sin \alpha
  - \binom{n}{3} \cos ^ {n - 3} \alpha \sin ^ {3} \alpha
  + \binom{n}{5} \cos ^ {n - 5} \alpha \sin ^ {5} \alpha
  \cdots $
\item
  $\sum_{n = 1}^{N} \cos nx = \frac{\sin(N + \frac{1}{2})x - \sin \frac{x}{2}}{2 \sin \frac{x}{2}}$
\item
  $\sum_{n = 1}^{N} \sin nx = \frac{-\cos(N + \frac{1}{2})x + \cos \frac{x}{2}}{2 \sin \frac{x}{2}}$
  % \item
  %   $e^x = 1 + x + \frac{x^2}{2!} + \frac{x^3}{3!} \cdots$ for $x \in (-\infty, +\infty)$ 
  % \item
  %   $\cos x = 1 - \frac{x^2}{2!} + \frac{x^4}{4!} \cdots$ for $x \in (-\infty, +\infty)$
  % \item
  %   $\sin x = x - \frac{x^3}{3!} + \frac{x^5}{5!} \cdots$ for $x \in (-\infty, +\infty)$
  % \item
  %   $\arcsin x = x + \sum_{n = 1}^{\infty} \frac{(2n - 1)!!}{(2n)!!} \frac{x^{2n + 1}}{2n + 1}$
  %   for $ x \in [-1, 1]$
  % \item
  %   $\arccos x = \frac{\pi}{2} - \sum_{n = 1}^{\infty} \frac{(2n - 1)!!}{(2n)!!} \frac{x^{2n + 1}}{2n + 1}$
  %   for $ x \in [-1, 1]$
  % \item
  %   $\arctan x = x - \frac{x^3}{3} + \frac{x^5}{5} \cdots$ for $x \in [-1, 1]$
  % \item
  %   $\ln(1 + x) = x - \frac{x^2}{2} + \frac{x^3}{3} \cdots$ for $x \in (-1, 1]$
\item
  $\int\limits_{0}^{\frac{\pi}{2}} \sin^n x {\rm d}x = 
  \left\{
    \begin{aligned}
      \frac{(n - 1)!!}{n!!} \times \frac{\pi}{2} &   & {n \textrm{是偶数}}\\
      \frac{(n - 1)!!}{n!!}                      &   & {n \textrm{是奇数}}
    \end{aligned}
  \right.
  $
\item
  $\int\limits_{0}^{+\infty} \frac{\sin x}{x} {\rm d}x = \frac{\pi}{2}$
\item
  $\int\limits_{0}^{+\infty} e^{-x^2} {\rm d}x = \frac{\sqrt{\pi}}{2}$
\item 傅里叶级数: 设周期为 $2T$. 函数分段连续. 在不连续点的值为左右极限的平均数.
  \begin{itemize}
  \item $a_n = \frac{1}{T} \int\limits_{-T}^{T} f(x) \cos \frac{n\pi}{T} x {\rm d}x$
  \item $b_n = \frac{1}{T} \int\limits_{-T}^{T} f(x) \sin \frac{n\pi}{T} x {\rm d}x$
  \item $f(x) = \frac{a_0}{2} +
    \sum\limits_{n = 1}^{+\infty}(a_n \cos \frac{n\pi}{T} x + b_n \sin \frac{n\pi}{T}x)$
  \end{itemize}
\item Beta 函数: $B(p, q) = \int\limits_{0}^{1} x^{p - 1} (1 - x) ^ {q - 1} {\rm d}x$
  \begin{itemize}
  \item 定义域 $(0, +\infty) \times (0, +\infty)$, 在定义域上连续
  \item $B(p, q) = B(q, p) = \frac{q - 1}{p + q - 1} B(p, q - 1)
    = 2\int\limits_{0}^{\frac{\pi}{2}} \cos^{2p - 1}\phi \sin^{2p - 1} \phi {\rm d}\phi
    = \int\limits_{0}^{+\infty} \frac{t^{q - 1}}{(1 + t)^{p + q}} {\rm d}t
    = \int\limits_{0}^{1} \frac{t^{p - 1} + t ^ {q - 1}}{(1 + t) ^ (p + q)}$
  \item $B(\frac{1}{2}, \frac{1}{2}) = \pi$
  \end{itemize}
\item Gamma 函数: $\Gamma = \int\limits_{0}^{+\infty} x^{s - 1} e ^ {-x} {\rm d}x$
  \begin{itemize}
  \item 定义域 $(0, +\infty)$, 在定义域上连续
  \item $\Gamma(1) = 1$, $\Gamma(\frac{1}{2}) = \sqrt{\pi}$
  \item $\Gamma(s) = (s - 1) \Gamma(s - 1)$
  \item $B(p, q) = \frac{\Gamma(p)\Gamma(q)}{\Gamma(p + q)}$
  \item $\Gamma(s)\Gamma(1 - s) = \frac{\pi}{\sin \pi s}$ for $s > 0$
  \item $\Gamma(s)\Gamma(s + \frac{1}{2}) = 2\sqrt{\pi} \frac{\Gamma(s)}{2^{2s - 1}}$ for $0 < s < 1$
  \end{itemize}
\item 积分:平面图形面积、曲线弧长、旋转体体积、旋转曲面面积
  $y = f(x)$,
  $\int\limits_{a}^{b}f(x){\rm d}x$,
  $\int\limits_{a}^{b} \sqrt{1 + f^{\prime 2}(x)} {\rm d}x$,
  $\pi\int\limits_{a}^{b} f^2(x) {\rm d}x$,
  $2\pi\int\limits_{a}^{b}|f(x)|\sqrt{1 + f^{\prime 2}(x)} {\rm d}x$

  $x = x(t), y = y(t), t \in [T_1, T_2]$,
  $\int\limits_{T_1}^{T_2} | y(t) x^\prime(t) | {\rm d}t$,
  $\int\limits_{T_1}^{T_2} \sqrt{x^{\prime 2}(t) + y^{\prime 2}(t)} {\rm d}t$,
  $\pi\int\limits_{T_1}^{T_2} |x^\prime(t)| y^2(t) {\rm d}t$,
  $2\pi\int\limits_{T_1}^{T_2}|y(t)|\sqrt{x^{\prime 2}(t) + y^{\prime 2}(t)}{\rm d}t$,

  $r = r(\theta), \theta \in [\alpha, \beta]$,
  $\frac{1}{2}\int\limits_{\alpha}^{\beta}r^2(\theta) {\rm d}\theta $,
  $\int\limits_{\alpha}^{\beta} \sqrt{r^{2}(\theta) + r^{\prime 2}(\theta)} {\rm d}\theta$,
  $\frac{2}{3}\pi\int\limits_{\alpha}^{\beta}r^3(\theta)\sin\theta{\rm d}\theta$,
  $2\pi\int\limits_{\alpha}^{\beta}r(\theta)\sin\theta\sqrt{r^2(\theta)+r^{\prime2}(\theta)}{\rm d}\theta$

\end{itemize}

\subsubsection{三次方程求根公式}
对一元三次方程
$x ^ 3 + px + q = 0$,
令
\begin{align*}
  A &= \sqrt[3]{-\frac{q}{2}+\sqrt{(\frac{q}{2})^2+(\frac{p}{3})^3}} \\
  B &= \sqrt[3]{-\frac{q}{2}-\sqrt{(\frac{q}{2})^2+(\frac{p}{3})^3}} \\ 
  \omega &= \frac{(-1 + \mathrm{i} \sqrt{3})}{2}
\end{align*}

则 $x_j = A\omega^{j} + B\omega^{2j}$ (j = 0, 1, 2).

当求解 $ax ^ 3 + bx ^ 2 + cx + d = 0$ 时, 令$x = y - \frac{b}{3a}$, 再求解$y$, 即转化为$y^3 + py + q = 0$ 的形式. 
其中, 
\begin{align*}
  p &= \frac{b^2 - 3ac}{3a^2} \\
  q &= \frac{2b ^ 3 - 9 abc + 27 a ^ 2 d}{27 a ^ 3}
\end{align*}

卡尔丹判别法: 
令$\Delta = (\frac{q}{2}) ^ 2 + (\frac{p}{3}) ^ 3$. 
当$\Delta > 0$时, 有一个实根和一对个共轭虚根;
当$\Delta = 0$时, 有三个实根, 其中两个相等;
当$\Delta < 0$时, 有三个不相等的实根. 

\subsubsection{椭圆}
\begin{itemize}
\item 椭圆$\frac{x^2}{a^2} + \frac{y^2}{b^2} = 1$, 其中离心率$e = \frac{c}{a}, c = \sqrt{a^2 - b^2}$; 焦点参数$p = \frac{b^2}{a}$
\item 椭圆上$(x, y)$点处的曲率半径为$R = a^2 b^2 (\dfrac{x^2}{a^4} + \dfrac{y^2}{b^4})^\frac{3}{2} = \dfrac{(r_1 r_2)^\frac{3}{2}}{ab}$, 其中$r_1$和$r_2$分别为$(x, y)$与两焦点$F_1$和$F_2$的距离. %设点$A$和点$M$的坐标分别为$(a, 0)$和$(x, y)$, 则$AM$的弧长为
  \[ L_{AM} = a \int_0^{\arccos{\frac{x}{a} }} \sqrt{1 - e^2 \cos^2 t} \textrm{d} t = a \int_{\arccos{\frac{x}{a} }}^\frac{\pi}{2} \sqrt{1 - e^2 \sin^2 t} \textrm{d} t\]
\item 椭圆的周长$L = 4a \int_0^{\frac{\pi}{2}} \sqrt{1 - e^2 \sin^2 t } \textrm{d} t = 4a E(e, \frac{\pi}{2})$, 其中
  \[ E(e, \frac{\pi}{2}) = \frac{\pi}{2} [ 1 - (\frac{1}{2})^2 e^2 - (\frac{1 \times 3}{2 \times 4})^2 \frac{e^4}{3} - (\frac{1 \times 3 \times 5}{2 \times 4 \times 6})^2 \frac{e^6}{5} - \cdots\]
\item 设椭圆上点$M(x, y), N(x, -y), x, y > 0, A(a, 0)$, 原点$O(0, 0)$, 扇形$OAM$的面积$S_{OAM} = \frac{1}{2} ab \arccos{\frac{x}{a}}$, 弓形$MAN$的面积$S_{MAN} = ab \arccos{\frac{x}{a}} - xy$.
\item 需要$5$个点才能确定一个圆锥曲线.
\item 设$\theta$为$(x, y)$点关于椭圆中心的极角, $r$为$(x, y)$到椭圆中心的距离, 椭圆极坐标方程:
  \[ x = r \cos \theta, y = r \sin \theta, r^2 = \frac{b^2 a^2}{b^2 \cos^2 \theta + a^2 \sin^2 \theta}\]
\end{itemize}

\subsubsection{抛物线}
\begin{itemize}
\item 标准方程$y^2 = 2px$, 曲率半径$ R = \dfrac{(p + 2x)^{\frac{3}{2} }}{\sqrt{p}}$
\item 弧长: 设$M(x, y)$是抛物线上一点, 则$L_{OM} = \frac{p}{2} [ \sqrt{\frac{2x}{p}(1 + \frac{2x}{p})} + \ln(\sqrt{\frac{2x}{p}} + \sqrt{1 + \frac{2x}{p}})]$
\item 弓形面积: 设$M, D$是抛物线上两点, 且分居一, 四象限. 做一条平行于$MD$且与抛物线相切的直线$L$. 若$M$到$L$的距离为$h$. 则有$S_{MOD} = \frac{2}{3}MD \cdot h$.
\end{itemize}

\subsubsection{重心}
\begin{itemize}
\item 半径$r$, 圆心角为$\theta$的扇形的重心与圆心的距离为$\dfrac{4r\sin\frac{\theta}{2}}{3\theta}$
\item 半径$r$, 圆心角为$\theta$的圆弧的重心与圆心的距离为$\dfrac{4r\sin^3\frac{\theta}{2}}{3(\theta - \sin\theta)}$
\item 椭圆上半部分的重心与圆心的距离为$\dfrac{4b}{3\pi}$
\item  抛物线中弓形$MOD$的重心满足$CQ = \frac{2}{5} PQ$, $P$是直线$L$与抛物线的切点, $Q$在$MD$上且$PQ$平行$x$轴, $C$是重心
\end{itemize}

\subsubsection{向量恒等式}
\begin{itemize}
\item $\overrightarrow{a} \cdot (\overrightarrow{b} \times \overrightarrow{c}) = \overrightarrow{b} \cdot (\overrightarrow{c} \times \overrightarrow{a}) = \overrightarrow{c} \cdot (\overrightarrow{a} \times \overrightarrow{b})$
\item $\overrightarrow{a} \times (\overrightarrow{b} \times \overrightarrow{c}) = (\overrightarrow{c} \times \overrightarrow{b}) \times \overrightarrow{a} = \overrightarrow{b}(\overrightarrow{a} \cdot \overrightarrow{c}) - \overrightarrow{c}(\overrightarrow{a} \cdot \overrightarrow{b})$
\end{itemize}

\subsubsection{常用几何公式}
\begin{itemize}
\item 三角形的五心
  \begin{itemize}
  \item 重心 $\overrightarrow{G} = \frac{\overrightarrow{A} + \overrightarrow{B} + \overrightarrow{C}}{3}$
  \item
    内心 $\overrightarrow{I} = \frac{a\overrightarrow{A} + b\overrightarrow{B} + c\overrightarrow{C}}{a + b + c}$,
    $R = \frac{2S}{a + b + c}$
  \item
    外心
    $x = \frac{\overrightarrow{A} + \overrightarrow{B} - \frac{\overrightarrow{BC} \cdot \overrightarrow{AC}}{\overrightarrow{AB} \times \overrightarrow{BC}} \overrightarrow{AB}^{T}}{2}$,
    $y = \frac{\overrightarrow{A} + \overrightarrow{B} + \frac{\overrightarrow{BC} \cdot \overrightarrow{AC}}{\overrightarrow{AB} \times \overrightarrow{BC}} \overrightarrow{AB}^{T}}{2}$,
    $R = \frac{abc}{4S}$
  \item
    垂心 $\overrightarrow{H} = 3\overrightarrow{G} - 2\overrightarrow{O}$
  \item
    旁心(三个) $\frac{-a\overrightarrow{A} + b\overrightarrow{B} + c\overrightarrow{C}}{-a + b + c}$
  \end{itemize}

\item 四边形: 设$D_1, D_2$为对角线, $M$为对角线中点连线, $A$为对角线夹角
  \begin{itemize}
  \item $a^2 + b^2 + c^2 + d^2 = D_1^2 + D_2 ^ 2 + 4 M^2$
  \item $S = \frac{1}{2} D_1  D_2 \sin A$
  \item $ac + bd = D_1 D_2$ (内接四边形适用)
  \item Bretschneider公式:
    $S = \sqrt{(p - a)(p - b)(p - c)(p - d) - abcd \cos^2(\frac{\theta}{2})}$, 其中 $\theta$ 为对角和
  \end{itemize}
\item 棱锥:
  \begin{itemize}
  \item 体积 $V = \frac{1}{3}Ah$, $A$为底面积, $h$ 为高
  \item (对正棱锥) 侧面积 $S = \frac{1}{2} lp $, $l$为斜高, $p$ 为底面周长
  \end{itemize}
\item 棱台:
  \begin{itemize}
  \item 体积$V = \dfrac{(A_1 + A_2 + \sqrt{A_1 A_2}) \cdot h}{3}$, $A_1$, $A_2$分别为上下底面面积, $h$为高
  \item (对正棱台)侧面积$S = \frac{1}{2} (p_1 + p_2) \cdot l$, $p_1$, $p_2$为上下底面周长, $l$为斜高. 
  \end{itemize}
\end{itemize}

\subsubsection{树的计数}
\begin{itemize}
\item 有根数计数: 令$S_{n, j} = \sum\limits_{1 \le i \le n / j} a_{n + 1 - ij} = S_{n - j, j} + a_{n + 1 - j}$\\
  于是, $n + 1$个结点的有根数的总数为$a_{n + 1} = \dfrac{\sum\limits_{1 \le j \le n} j \cdot a_j \cdot S_{n, j} }{n}$\\
  附: $a_1 = 1, a_2 = 1, a_3 = 2, a_4 = 4, a_5 = 9, a_6 = 20, a_9 = 286, a_{11} = 1842$
\item 无根树计数: 当$n$是奇数时, 则有$a_n - \sum\limits_{1 \le i \le \frac{n}{2}} a_i a_{n - i}$种不同的无根树\\
  当$n$是偶数时, 则有$a_n - \sum\limits_{1 \le i \le \frac{n}{2}} a_i a_{n - i} + \dfrac{1}{2} a_\frac{n}{2} (a_\frac{n}{2} + 1)$种不同的无根树
\item Matrix-Tree定理: 对任意图$G$, 设mat[$i$][$i$] = $i$的度数, mat[$i$][$j$] = $i$与$j$之间边数的相反数, 则mat[$i$][$j$]的任意余子式的行列式就是该图的生成树个数
\end{itemize}

\subsection{小知识}
\begin{itemize}
\item 勾股数: 设正整数$n$的质因数分解为$n = \prod p_i ^ {a_i}$, 
  则$x^2+y^2=n$有整数解的充要条件是$n$中不存在形如$p_i \equiv 3\pmod{4}$且指数$a_i$为奇数的质因数$p_i$.
  $(\frac{a - b}{2})^2 + ab = (\frac{a + b}{2})^2$.
\item 素勾股数: 若 $m$ 和 $n$ 互质, 而且 $m$ 和 $n$ 中有一个是偶数, 则$a = m^2 - n^2$, $b = 2mn$, $c = m^2 + n^2$, 则$a$、$b$、$c$是素勾股数. 
\item Stirling公式: $n! \approx \sqrt{2 \pi n} (\frac{n}{e})^n$
\item Pick定理: 简单多边形, 不自交, 顶点如果全是整点. 则: 
  $ \textrm{严格在多边形内部的整点数} + \frac{1}{2} \textrm{在边上的整点数} - 1 = \textrm{面积}$
\item Mersenne素数: $p$是素数且$2^p-1$的数是素数. (10000以内的$p$有: 2, 3, 5, 7, 13, 17, 19, 31, 61, 89, 107, 127, 521, 607, 1279, 2203, 2281, 3217, 4253, 4423, 9689, 9941)
\item 序列差分表: 差分表的第$0$条对角线确定原序列. 
  设原序列为$h_i$, 第$0$条对角线为$c_0,c_1,\ldots,c_p,0,0,\ldots$. 
  有这样两个公式: 
  $h_n = \binom{n}{0}c_0 + \binom{n}{1}c_1 + \ldots + \binom{n}{p} c_p$, 
  $\sum_{k = 0}^{n}h_k = \binom{n+1}{1}c_0 + \binom{n+1}{2}c_2 + \ldots + \binom{n+1}{p+1}c_p$
\item GCD:
  $\gcd(2^a-1,2^b-1)=2^{\gcd(a,b)}-1$
\item Fermat分解算法: 
  从$t=\sqrt{n}$开始, 
  依次检查$t^2-n,(t+1)^2-n,(t+2)^2-n,\ldots$, 
  直到出现一个平方数$y$, 
  由于$t ^ 2 - y ^ 2 = n$, 
  因此分解得$n = (t -y)(t + y)$. 
  显然, 当两个因数很接近时这个方法能很快找到结果, 
  但如果遇到一个素数, 则需要检查$\frac{n + 1}{2} - \sqrt{n}$个整数
\item 牛顿迭代:
  $x_1 = x_0 - \frac{f(x_0)}{f^\prime(x_0)}$
\item 球与盒子的动人故事: ($n$个球, $m$个盒子, $S$为第二类斯特林数)
  \begin{enumerate}
  \item 球同, 盒同, 无空: dp
  \item 球同, 盒同, 可空: dp
  \item 球同, 盒不同, 无空: $\binom{n - 1}{m - 1}$
  \item 球同, 盒不同, 可空: $\binom{n + m - 1}{n - 1}$
  \item 球不同, 盒同, 无空: $S(n, m)$
  \item 球不同, 盒同, 可空: $\sum_{k = 1}^{m} S(n, k)$
  \item 球不同, 盒不同, 无空: $m! S(n, m)$
  \item 球不同, 盒不同, 可空: $m^n$
  \end{enumerate}
\item 组合数奇偶性: 若 $(n \& m) = m$, 则 $\binom{n}{m}$ 为奇数, 否则为偶数
\item 格雷码 $G(x) = x \otimes (x >> 1) $
\item Fibonacci数: 
  \begin{itemize}
  \item $F_0 = F_1 = 1$, $F_i = F_{i - 1} + F_{i - 2}$, $F_{-i} = (-1) ^ {i - 1} F_i$
  \item $F_i = \cfrac{1}{\sqrt{5}} ((\cfrac{1 + \sqrt{5}}{2}) ^ n - (\cfrac{1 - \sqrt{5}}{2}) ^ {n}) $
  \item $\gcd(F_n,F_m)=F_{\gcd(n,m)}$
  \item $F_{i + 1} F_i - F_i^2 = (-1) ^ i$
  \item $F_{n + k} = F_k F_{n + 1} + F_{k - 1} F_n$
  \end{itemize}
\item 第一类 Stirling 数: $\stlf{n}{k}$ 代表第一类无符号 Stirling 数, 代表将 $n$ 阶置换群中有 $k$ 个环的置换个数; $s(n,k)$代表有符号型, $s(n, k) = (-1)^{n - k}\stlf{n}{k}$.
  \begin{itemize}
  \item $(x)^{(n)} = \sum\limits_{k = 0}^{n}\stlf{n}{k}x ^k$, $(x)_{n} = \sum\limits_{k = 0}^{n} s(n, k) x ^k$
  \item $\stlf{n}{k} = n\stlf{n - 1}{k} + \stlf{n - 1}{k - 1}$, $\stlf{0}{0} = 1$, $\stlf{n}{0} = \stlf{0}{n} = 0$
  \item $\stlf{n}{n - 2} = \frac{1}{4} (3n - 1) \binom{n}{3} $, $\stlf{n}{n - 3} = \binom{n}{2} \binom{n}{4} $
  \item $\sum\limits_{k = 0}^{a}\stlf{n}{k} = n! - \sum\limits_{k = 0}^{n} \stlf{n}{k + a + 1}$
  \item $\sum\limits_{p = k}^{n}\stlf{n}{p}\binom{p}{k} = \stlf{n + 1}{k + 1}$
    % \item $s(n, n - p) = \frac{1}{(n - p - 1)!} \sum\limits_{0 \le k1, k2, \ldots, k_p: \sum}^{n}$
  \end{itemize}
\item 第二类 Stirling 数: $\stls{n}{k} = S(n, k)$ 代表 $n$个不同的球, 放到 $k$ 个相同的盒子里, 盒子非空.
  \begin{itemize}
  \item $\stls{n}{k} = \frac{1}{k!} \sum\limits_{j = 0}^{k} (-1)^j \binom{k}{j} (k - j)^n$
  \item $\stls{n + 1}{k} = k\stls{n}{k} + \stls{n}{k - 1}$, $\stls{0}{0} = 1$, $\stls{n}{0} = \stls{0}{n} = 0$
  \item 奇偶性: $(n - k) \& \frac{k - 1}{2} == 0$
  \end{itemize}
\item Bell 数: $B_n$ 代表将 $n$ 个元素划分成若干个非空集合的方案数
  \begin{itemize}
  \item $B_0 = B_1 = 1$, $B_n = \sum\limits_{k = 0}^{n - 1} \binom{n - 1}{k} B_k$
  \item $B_n = \sum\limits_{k = 0}^{n} \stls{n}{k} $
  \item Bell 三角形: $a_{1, 1} = 1$, $a_{n, 1} = a_{n - 1, n - 1}$, $a_{n, m} = a_{n, m - 1} + a_{n - 1, m - 1}$, $B_n = a_{n, 1}$
  \item 对质数$p$, $B_{n + p} \equiv B_n + B_{n + 1} \pmod{p}$
  \item 对质数$p$, $B_{n + p^m} \equiv mB_n + B_{n + 1} \pmod{p}$
  \item 对质数$p$, 模的周期一定是 $\frac{p^p - 1}{p - 1}$ 的约数, $p \le 101$时就是这个值
  \item 从$B_0$开始, 前几项是 $1, 1, 2, 5, 15, 52, 203, 877, 4140, 21147, 115975 \cdots$
  \end{itemize}
\item Bernoulli 数
  \begin{itemize}
  \item $B_0 = 1$, $B_1 = \frac{1}{2}$, $B_2 = \frac{1}{6}$, $B_4 = -\frac{1}{30}$, $B_6 = \frac{1}{42}$, $B_8 = B_4$, $B_{10} = \frac{5}{66}$
  \item $\sum\limits_{k = 1}^{n} k^m = \cfrac{1}{m + 1} \sum\limits_{k = 0}^{m} \binom{m + 1}{k} B_k n ^ {m + 1 - k} $
  \item $B_m = 1 - \sum\limits_{k = 0}^{m - 1} \binom{m}{k} \frac{B_k}{m - k + 1}$
  \end{itemize}
\item 完全数: $x$ 是偶完全数等价于 $x = 2^{n - 1} (2^n - 1)$, 且 $2^n - 1$ 是质数.
\end{itemize}

\section{其他}
\subsection{Extended LIS}
\lstinputlisting{"Others/K Disjoint LIS.cpp"}

\subsection{生成 nCk}
\lstinputlisting{"Others/next nCk.cpp"}

\subsection{nextPermutation}
\lstinputlisting[language=Java]{"Others/nextPermutation.java"}

\subsection{Josephus 数与逆 Josephus 数}
\lstinputlisting{"Others/Josephus.cpp"}

\subsection{表达式求值}
\lstinputlisting{"Others/Expression Evaluation.cpp"}

\subsection{曼哈顿最小生成树}
\lstinputlisting{"Others/Manhattan MST.cpp"}

\subsection{直线下的整点个数}
求 $\sum_{i=0}^{n-1} \lfloor\frac{a+bi}{m}\rfloor$
\lstinputlisting{"Others/DotsBelowLine.cpp"}

\subsection{Java多项式}
\lstinputlisting[language=Java]{"Others/Java Polynomial.java"}

\subsection{long long 乘法取模}
\lstinputlisting{"Others/Long Long Multi and Mod.cpp"}

\subsection{重复覆盖}
\lstinputlisting{"Others/Multiple Cover - Full.cpp"}

\subsection{星期几判定}
\lstinputlisting{"Others/Zeller.cpp"}

\subsection{LCSequence Fast}
\lstinputlisting{"Others/LCSFast.cpp"}

\subsection{C Split}
\lstinputlisting{"Others/C Split.cpp"}

\subsection{builtin系列}
\begin{itemize}
\item int \_\_builtin\_ffs (unsigned int x) 返回x的最后一位1的是从后向前第几位,  比如7368( 1110011001000) 返回4. 
\item int \_\_builtin\_clz (unsigned int x) 返回前导的0的个数. 
\item int \_\_builtin\_ctz (unsigned int x) 返回后面的0个个数, 和\_\_builtin\_clz相对. 
\item int \_\_builtin\_popcount (unsigned int x) 返回二进制表示中1的个数. 
\item int \_\_builtin\_parity (unsigned int x) 返回x的奇偶校验位, 也就是x的1的个数模2的结果. 
\end{itemize}

\section{Templates}
\subsection{泰勒级数}
{\allowdisplaybreaks
\begin{align*}
\cfrac{1}{1 - x}  &=  1 + x + x^2 + x^3 + x^4 + \cdots    &=  \sum_{i=0}^\infty x^i \\
\cfrac{1}{1 - c x}  &=  1 + c x + c^2 x^2 + c^3 x^3 + \cdots  &=  \sum_{i=0}^\infty c^i x^i \\
\cfrac{1}{1 - x^n}  &=  1 + x^n + x^{2n} + x^{3n} + \cdots    &=  \sum_{i=0}^\infty x^{ni} \\
\cfrac{x}{(1 - x)^2}&=  x + 2 x^2 + 3 x^3 + 4 x^4 + \cdots    &=  \sum_{i=0}^\infty i x^i \\
\sum_{k=0}^n \stls{n}{k} \cfrac{k! z^k}{(1-z)^{k+1}}
          &=  x + 2^n x^2 + 3^n x^3 + 4^n x^4 + \cdots
                                &=  \sum_{i=0}^\infty i^n x^i \\
e^x          &= 1 + x + \cfrac{1}{2} x^2 + \cfrac{1}{6} x^3 + \cdots
                                &=  \sum_{i=0}^\infty \cfrac{x^i}{i!} \\
\ln (1 + x)      &=  x - \cfrac{1}{2} x^2 + \cfrac{1}{3} x^3 - \cfrac{1}{4} x^4  - \cdots
                                &=  \sum_{i=1}^\infty (-1)^{i+1} \cfrac{x^i}{i} \\
\ln \cfrac{1}{1 - x}
          &=  x + \cfrac{1}{2} x^2 + \cfrac{1}{3} x^3 + \cfrac{1}{4} x^4  + \cdots
                                &=  \sum_{i=1}^\infty \cfrac{x^i}{i} \\
\sin x        &=  x - \cfrac{1}{3!} x^3  + \cfrac{1}{5!} x^5 - \cfrac{1}{7!} x^7 +  \cdots
                                &=  \sum_{i=0}^\infty (-1)^i \cfrac{x^{2i+1}}{(2i+1)!} \\
\cos x        &=  1 - \cfrac{1}{2!} x^2  + \cfrac{1}{4!} x^4 - \cfrac{1}{6!} x^6 +  \cdots
                                &=  \sum_{i=0}^\infty (-1)^i \cfrac{x^{2i}}{(2i)!} \\
\tan^{-1} x      &=  x - \cfrac{1}{3} x^3  + \cfrac{1}{5} x^5 - \cfrac{1}{7} x^7 +  \cdots
                                &=  \sum_{i=0}^\infty (-1)^i \cfrac{x^{2i+1}}{(2i+1)} \\
(1+x)^n        &=  1 + n x + \cfrac{n(n-1)}{2} x^2 +  \cdots
                                &=  \sum_{i=0}^\infty {n \choose i} x^i \\
\cfrac{1}{(1-x)^{n+1}}
          &=  1 + (n+1) x + {n+2 \choose 2} x^2 +  \cdots
                                &=  \sum_{i=0}^\infty {i+ n \choose i} x^i \\
\cfrac{x}{e^x - 1}  &=  1 - \cfrac{1}{2} x + \cfrac{1}{12} x^2 - \cfrac{1}{720} x^4 + \cdots
                                &=  \sum_{i=0}^\infty \cfrac{B_i x^i}{i!} \\
\cfrac{1}{2x}(1 - \sqrt{1-4x})
          &=  1 + x + 2 x^2 + 5 x^3 +  \cdots
                                &=  \sum_{i=0}^\infty \cfrac{1}{i+1}{2i \choose i}x^i \\
\cfrac{1}{\sqrt{1-4x}}
          &=  1 + 2 x + 6 x^2 + 20 x^3 + \cdots
                                &=  \sum_{i=0}^\infty {2i \choose i}x^i \\
\cfrac{1}{\sqrt{1-4x}}\left(\cfrac{1 - \sqrt{1-4x}}{2x}\right)^n
          &=  1 + (2+n)x + {4+n \choose 2} x^2 + \cdots
                                &=  \sum_{i=0}^\infty {2i+n \choose i}x^i \\
\cfrac{1}{1-x}\ln\cfrac{1}{1 - x}
          &=  x + \cfrac{3}{2} x^2 + \cfrac{11}{6} x^3 + \cfrac{25}{12} x^4 + \cdots
                                &=  \sum_{i=1}^\infty H_i x^i \\
\cfrac{1}{2}\left(\ln\cfrac{1}{1- x}\right)^2
          &=  \cfrac{1}{2} x^2 + \cfrac{3}{4} x^3 + \cfrac{11}{24} x^4 + \cdots
                                &=  \sum_{i=2}^\infty \cfrac{H_{i-1} x^i}{i} \\
\cfrac{x}{1 - x - x^2}
          &=  x + x^2 + 2 x^3 + 3 x^4 +  \cdots
                                &=  \sum_{i=0}^\infty F_i x^i \\
\cfrac{F_n x}{1 - (F_{n-1} + F_{n+1})x - (-1)^n x^2}
          &=  F_n x + F_{2n} x^2 + F_{3n} x^3 + \cdots
                                &=  \sum_{i=0}^\infty F_{ni} x^i
\end{align*}
}


\subsection{积分表}
\def\du\dx{{ \cfrac{ \text{d}u }{ \text{d}x } }}
\def\d{{ \text{d} }}
\def\dx{{ \d x}}
\def\du{{ \d u }}
\def\dv{{ \d v }}

\def\sech{{ \mathop{\text{sech}} }}
\def\csch{{ \mathop{\text{csch}} }}

\def\arcsin{{ \mathop{\text{arcsin}} }}
\def\arccos{{ \mathop{\text{arccos}} }}
\def\arctan{{ \mathop{\text{arctan}} }}
\def\arccot{{ \mathop{\text{arccot}} }}
\def\arcsec{{ \mathop{\text{arcsec}} }}
\def\arccsc{{ \mathop{\text{arccsc}} }}

\def\arcsinh{{ \mathop{\text{arcsinh}} }}
\def\arccosh{{ \mathop{\text{arccosh}} }}
\def\arctanh{{ \mathop{\text{arctanh}} }}
\def\arccoth{{ \mathop{\text{arccoth}} }}
\def\arcsech{{ \mathop{\text{arcsech}} }}
\def\arccsch{{ \mathop{\text{arccsch}} }}

\begin{itemize}
\item $ \displaystyle \d(\tan x) = \sec^2 x \dx $
\item $ \displaystyle \d(\cot x) = \csc^2 x \dx $
\item $ \displaystyle \d(\sec x) = \tan x \, \sec x \dx $
\item $ \displaystyle \d(\csc x) = -\cot x \, \csc x \dx $
\item $ \displaystyle d(\arcsin x) = \cfrac{1}{\sqrt{1 - x^2}} \dx $
\item $ \displaystyle d(\arccos x) = \cfrac{-1}{\sqrt{1 - x^2}} \dx $
\item $ \displaystyle d(\arctan x) = \cfrac{1}{1 + x^2} \dx $
\item $ \displaystyle d(\arccot x) = \cfrac{-1}{1 + x^2} \dx $
\item $ \displaystyle d(\arcsec x) = \cfrac{1}{x \sqrt{1 - x^2}}\dx $
\item $ \displaystyle d(\arccsc x) = \cfrac{-1}{u \sqrt{1 - x^2}}\dx $
\end{itemize}

\begin{itemize}
\item $ \displaystyle \int c u \,\dx = c \int u \,\dx$
\item $ \displaystyle \int (u + v) \,\dx = \int u \,\dx + \int v \,\dx$
\item $ \displaystyle \int x^n \,\dx = \cfrac{1}{n + 1}x^{n+1}, \quad n \neq -1$
\item $ \displaystyle \int \cfrac{1}{x}\dx = \ln x$
\item $ \displaystyle \int e^x \,\dx = e^x$
\item $ \displaystyle \int \cfrac{\dx}{1 + x ^ 2} = \arctan x$
\item $ \displaystyle \int u \cfrac{\dv}{\dx} \dx = uv - \int v \cfrac{\du}{\dx} \dx$
\item $ \displaystyle \int \sin x \,\dx = -\cos x$
\item $ \displaystyle \int \cos x \,\dx = \sin x$
\item $ \displaystyle \int \tan x \,\dx = -\ln \vert \cos x \vert$
\item $ \displaystyle \int \cot x \,\dx = \ln \vert \cos x \vert$
\item $ \displaystyle \int \sec x \,\dx = \ln \vert \sec x + \tan x \vert$
\item $ \displaystyle \int \csc x \,\dx = \ln \vert \csc x + \cot x \vert$
\item $ \displaystyle \int \arcsin \cfrac{x}{a} \dx = \arcsin \cfrac{x}{a} + \sqrt{a^2 - x^2}, \quad \hbox{$a > 0$}$
\item $ \displaystyle \int \arccos \cfrac{x}{a} \dx = \arccos \cfrac{x}{a} - \sqrt{a^2 - x^2}, \quad \hbox{$a > 0$}$
\item $ \displaystyle \int \arctan \cfrac{x}{a} \dx = x \arctan \cfrac{x}{a} - \cfrac{a}{2} \ln(a^2 + x^2), \quad \hbox{$a > 0$}$
\item $ \displaystyle \int \sin^2 (a x) \dx = \cfrac{1}{2a} \big(ax - \sin(ax) \cos(ax)\big)$
\item $ \displaystyle \int \cos^2 (a x) \dx = \cfrac{1}{2a} \big(ax + \sin(ax) \cos(ax)\big)$
\item $ \displaystyle \int \sec^2 x \, \dx = \tan x$
\item $ \displaystyle \int \csc^2 x \, \dx = - \cot x$
\item $ \displaystyle \int \sin^n x \, \dx = -\cfrac{\sin^{n-1} x \cos x}{n} + \cfrac{n-1}{n}\int \sin^{n-2} x \, \dx$
\item $ \displaystyle \int \cos^n x \, \dx = \cfrac{\cos^{n-1} x \sin x}{n} + \cfrac{n-1}{n}\int \cos^{n-2} x \, \dx$
\item $ \displaystyle \int \tan^n x \, \dx = \cfrac{\tan^{n-1} x }{n - 1} - \int \tan^{n-2} x \, \dx,\quad n \neq 1$
\item $ \displaystyle \int \cot^n x \, \dx = -\cfrac{\cot^{n-1} x}{n - 1} - \int \cot^{n-2} x \, \dx,\quad n \neq 1$
\item $ \displaystyle \int \sec^n x \, \dx = \cfrac{\tan x\sec^{n-1} x}{n - 1} + \cfrac{n-2}{n-1}\int \sec^{n-2} x \, \dx,\quad n \neq 1$
\item $ \displaystyle \int \csc^n x \, \dx = -\cfrac{\cot x\csc^{n-1} x}{n - 1} + \cfrac{n-2}{n-1}\int \csc^{n-2} x \, \dx,\quad n \neq 1$
\item $ \displaystyle \int \sinh x \, \dx = \cosh x$
\item $ \displaystyle \int \cosh x \, \dx = \sinh x$
\item $ \displaystyle \int \tanh x \, \dx = \ln \vert \cosh x \vert$
\item $ \displaystyle \int \coth x \, \dx = \ln \vert \sinh x \vert$
\item $ \displaystyle \int \sech x \, \dx = \arctan \sinh x $
\item $ \displaystyle \int \csch x \, \dx = \ln\left\vert\tanh \cfrac{x}{2} \right\vert$
\item $ \displaystyle \int \sinh^2 x \, \dx = \cfrac{1}{4} \sinh (2x) - \cfrac{1}{2} x$
\item $ \displaystyle \int \cosh^2 x \, \dx = \cfrac{1}{4} \sinh (2x) + \cfrac{1}{2} x$
\item $ \displaystyle \int \sech^2 x \, \dx = \tanh x$
\item $ \displaystyle \int \arcsinh \cfrac{x}{a} \dx = x \arcsinh \cfrac{x}{a} - \sqrt{x^2 + a^2},\quad a > 0$
\item $ \displaystyle \int \arctanh \cfrac{x}{a} \dx = x \arctanh \cfrac{x}{a} + \cfrac{a}{2} \ln\vert a^2 - x^2\vert$
\item $ \displaystyle \int \arccosh \cfrac{x}{a} \dx = 
  \begin{cases}
  \displaystyle x \arccosh \cfrac{x}{a} - \sqrt{x^2 + a^2}, \text{if $\arccosh \cfrac{x}{a} > 0$ and $a > 0$} \\
  \displaystyle x \arccosh \cfrac{x}{a} + \sqrt{x^2 + a^2}, \text{if $\arccosh \cfrac{x}{a} < 0$ and $a > 0$}
  \end{cases} $

\item $ \displaystyle \int \cfrac{\dx}{\sqrt{a^2 + x^2}}= \ln \left(x + \sqrt{a^2 + x^2}\right),\quad a > 0 $
\item $ \displaystyle \int \cfrac{\dx}{a^2 + x^2}= \cfrac{1}{a} \arctan \cfrac{x}{a} ,\quad a > 0 $
\item $ \displaystyle \int \sqrt{a^2 - x^2} \, \dx = \cfrac{x}{2} \sqrt{a^2 - x^2} + \cfrac{a^2}{2} \arcsin \cfrac{x}{a} ,\quad a > 0 $
\item $ \displaystyle \int (a^2 - x^2)^{3/2} \dx = \cfrac{x}{8} (5a^2 - 2x^2)\sqrt{a^2 - x^2} + \cfrac{3a^4}{8} \arcsin \cfrac{x}{a} ,\quad a > 0 $
\item $ \displaystyle \int \cfrac{\dx}{\sqrt{a^2 - x^2}} = \arcsin \cfrac{x}{a} ,\quad a > 0 $
\item $ \displaystyle \int \cfrac{\dx}{a^2 - x^2} = \cfrac{1}{2a} \ln\left\vert \cfrac{a + x}{a - x}\right\vert $
\item $ \displaystyle \int \cfrac{\dx}{(a^2 - x^2)^{3/2}} = \cfrac{x}{a^2\sqrt{a^2 - x^2}} $
\item $ \displaystyle \int \sqrt{a^2 \pm x^2} \, \dx = \cfrac{x}{2} \sqrt{a^2 \pm x^2} \pm \cfrac{a^2}{2} \ln\left\vert x + \sqrt{a^2 \pm x^2}\right\vert $
\item $ \displaystyle \int \cfrac{\dx}{\sqrt{x^2 - a^2}}= \ln\left\vert x + \sqrt{x^2 - a^2}\right\vert, \quad a > 0 $
\item $ \displaystyle \int \cfrac{\dx}{a x^2 + b x}= \cfrac{1}{a} \ln\left\vert\cfrac{x}{a + bx}\right\vert $
\item $ \displaystyle \int x \sqrt{a + b x} \, \dx= \cfrac{2(3bx - 2a)(a + bx)^{3/2}}{15 b^2} $
\item $ \displaystyle \int \cfrac{\sqrt{a + b x}}{x} \, \dx= 2\sqrt{a + b x} + a \int \cfrac{1}{x \sqrt{a + b x}} \dx $
\item $ \displaystyle \int \cfrac{x}{\sqrt{a + b x}} \, \dx= \cfrac{1}{\sqrt{2}} \ln\left\vert\cfrac{\sqrt{a + b x} - \sqrt{a}}{\sqrt{a + b x} + \sqrt{a}}\right\vert, \quad a > 0 $
\item $ \displaystyle \int \cfrac{\sqrt{a^2 - x^2}}{x} \, \dx = \sqrt{a^2 - x^2} - a \ln\left\vert\cfrac{a + \sqrt{a^2 - x^2}}{x}\right\vert $
\item $ \displaystyle \int x \sqrt{a^2 - x^2} \, \dx = - \cfrac{1}{3} (a^2 - x^2)^{3/2} $
\item $ \displaystyle \int x^2 \sqrt{a^2 - x^2} \, \dx = \cfrac{x}{8} (2x^2 - a^2)\sqrt{a^2 - x^2} + \cfrac{a^4}{8} \arcsin \cfrac{x}{a} , \quad a > 0 $
\item $ \displaystyle \int \cfrac{\dx}{\sqrt{a^2 - x^2}}= - \cfrac{1}{a} \ln\left\vert\cfrac{a + \sqrt{a^2 - x^2}}{x}\right\vert $
\item $ \displaystyle \int \cfrac{x \, \dx}{\sqrt{a^2 - x^2}} = - \sqrt{a^2 - x^2} $
\item $ \displaystyle \int \cfrac{x^2 \, \dx}{\sqrt{a^2 - x^2}} = - \cfrac{x}{2} \sqrt{a^2 - x^2} + \cfrac{a^2}{2} \arcsin \cfrac{x}{a}, \quad a > 0 $
\item $ \displaystyle \int \cfrac{\sqrt{a^2 + x^2}}{x} \, \dx = \sqrt{a^2 + x^2} - a \ln\left\vert\cfrac{a + \sqrt{a^2 + x^2}}{x}\right\vert $
\item $ \displaystyle \int \cfrac{\sqrt{x^2 - a^2}}{x} \, \dx = \sqrt{x^2 - a^2} - a \arccos \cfrac{a}{\vert x\vert} ,\quad a > 0 $
\item $ \displaystyle \int x \sqrt{x^2 \pm a^2} \, \dx = \cfrac{1}{3} (x^2 \pm a^2)^{3/2} $
\item $ \displaystyle \int \cfrac{\dx}{x \sqrt{x^2 + a^2}} = \cfrac{1}{a} \ln \left\vert\cfrac{x}{a + \sqrt{a^2 + x^2}}\right\vert $

\item $ \displaystyle \int \cfrac{\dx}{x \sqrt{x^2 - a^2}} = \cfrac{1}{a} \arccos \cfrac{a}{\vert x\vert} ,\quad a > 0 $
\item $ \displaystyle \int \cfrac{\dx}{x^2 \sqrt{x^2 \pm a^2}} = \mp \cfrac{\sqrt{x^2 \pm a^2}}{a^2 x} $
\item $ \displaystyle \int \cfrac{x \, \dx}{\sqrt{x^2 \pm a^2}} = \sqrt{x^2 \pm a^2} $
\item $ \displaystyle \int \cfrac{\sqrt{x^2 \pm a^2}}{x^4} \, \dx = \mp \cfrac{(x^2 + a^2)^{3/2}}{3a^2 x^3} $
\item $ \displaystyle \int \cfrac{\dx}{a x^2  + bx + c} = 
  \begin{cases}
  \displaystyle \cfrac{1}{\sqrt{b^2 -4ac}} \ln \left\vert\cfrac{ 2ax + b - \sqrt{b^2 -4ac}}{2ax + b + \sqrt{b^2 -4ac}}\right\vert, \text{if $b^2 > 4ac$} \\
  \displaystyle \cfrac{2}{\sqrt{4ac - b^2}} \arctan \cfrac{ 2ax + b }{\sqrt{4ac - b^2}}, \text{if $b^2 < 4ac$}
  \end{cases}
$
\item $ \displaystyle \int \cfrac{\dx}{\sqrt{a x^2  + bx + c}}  =
  \begin{cases}
  \displaystyle \cfrac{1}{\sqrt{a}} \ln \left\vert 2ax + b + 2\sqrt{a} \sqrt{ax^2 + bx + c}\right\vert, \text{if $a > 0$} \\
  \displaystyle \cfrac{1}{\sqrt{- a}} \arcsin \cfrac{-2ax - b}{\sqrt{b^2 - 4ac}}, \text{if $a < 0$}
  \end{cases}
$

\item $ \displaystyle \int \sqrt{a x^2  + bx + c} \, \dx = \cfrac{2ax + b}{4a} \sqrt{a x^2  + bx + c} + \cfrac{4ax - b^2}{8a} \int \cfrac{\dx}{\sqrt{a x^2  + bx + c}} $
\item $ \displaystyle \int \cfrac{x \, \dx}{ \sqrt{a x^2  + bx + c}} = \cfrac{\sqrt{a x^2  + bx + c} }{ a} - \cfrac{b }{ 2a} \int \cfrac{\dx }{ \sqrt{a x^2  + bx + c}} $
\item $ \displaystyle \int \cfrac{\dx }{ x \sqrt{a x^2  + bx + c}} =
  \begin{cases}
  \displaystyle \cfrac{-1}{ \sqrt{c}} \ln \left\vert\cfrac{2\sqrt{c} \sqrt{ax^2 + bx + c} + bx + 2c }{ x}\right\vert, \text{if $c > 0$} \\
  \displaystyle \cfrac{1 }{ \sqrt{- c}} \arcsin \cfrac{bx + 2c }{ \vert x \vert\sqrt{b^2 - 4ac}}, \text{if $c < 0$}
  \end{cases}
$
\item $ \displaystyle \int x^3  \sqrt{x^2 + a^2} \, \dx = (\cfrac{1}{3} x^2 - \cfrac{2}{15} a^2)(x^2 + a^2)^{3/2} $
\item $ \displaystyle \int x^n \sin (ax) \, \dx = - \cfrac{1}{a} x^n \cos (ax) + \cfrac{n}{a} \int x^{n-1} \cos (ax) \, \dx $
\item $ \displaystyle \int x^n \cos (ax) \, \dx = \cfrac{1}{a} x^n \sin (ax) - \cfrac{n}{a} \int x^{n-1} \sin (ax) \, \dx $
\item $ \displaystyle \int x^n e^{ax} \, \dx = \cfrac{x^n e^{ax}}{a} - \cfrac{n}{a} \int x^{n-1} e^{ax} \, \dx $
\item $ \displaystyle \int x^n \ln (ax) \, \dx = x^{n+1}\left(\cfrac{\ln (ax) }{ n+1} - \cfrac{1 }{ (n+1)^2}\right) $
\item $ \displaystyle \int x^n (\ln ax)^m \, \dx = \cfrac{x^{n+1}}{n+1}(\ln ax)^m - \cfrac{m}{n+1}\int x^n (\ln ax)^{m-1} \, \dx $

\end{itemize}



\subsection{Eclipse配置}
Exec=env UBUNTU\_MENUPROXY= /opt/eclipse/eclipse

preference general keys 把 word completion 设置成 alt+c, 把 content assistant 设置成 alt + /

\subsection{C++}
\lstinputlisting{"Templates/cpp_template.cpp"}

\subsection{Java}
\lstinputlisting[language=Java]{"Templates/java_template.java"}

\subsection{gcc配置}
在 .bashrc 中加入 export CXXFLAGS="-Wall -Wconversion -Wextra -g3"
 
\end{document}`